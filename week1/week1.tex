\documentclass[UTF8]{ctexart}
\usepackage{xcolor}
\usepackage{graphicx}
\usepackage[margin=1.5in]{geometry}
\usepackage{float}
\usepackage{listings}
\usepackage{fancyhdr}


\title{Week 1}
\author{左昊天}
\date{\today}

\pagestyle{fancy}
\fancyhead[L]{ }
\setlength{\headheight}{27pt}

\lstset{
    numbers=left,
    numberstyle=\tiny,
    frame=shadowbox,
    rulesepcolor= \color{ red!20!green!20!blue!20},
    escapeinside=``,
    xleftmargin=2em,aboveskip=1em,
    framexleftmargin=2em,
    breaklines=true
}

\begin{document}

\maketitle

\section{Latex}

\subsection{章节}
章节:\verb|\section{}|

子章节:\verb|\subsection{}|

三级章节:\verb|\subsubsection{}|

\subsection{加粗}\verb|\textbf{}|
\subsection{斜体italic}
\verb|\textit{}|  
 
\textcolor{red}{注:中文的斜体和英文的斜体样式不一样}

例:\textit{中文样式}\qquad\textit{English version}

\subsection{下划线}\verb|\underline{}|

\subsection{添加图片}
先引入graphicx包,再用\verb|\includegraphics{}| 

\subsection{添加图片标题} 
先将图片放在figure中,再在图片的下面加\verb|\caption{}|
例:

\begin{figure}[H]
    \centering
    \includegraphics[width=0.2\textwidth]{photo.jpg}
    \caption{举例}
\end{figure}

\textbf{遇到的问题:}
\bigskip

(1)添加图片后,不显示图片

原因:图片过大导致无法显示

解决办法:通过\verb|\includegraphics[width=0.2\textwidth]{photo.jpg}|等比例设置图片的宽高。
\bigskip

(2)图片出现在了章节前面。

原因:figure是浮动体,系统会自动决定浮动体的放置位置

解决办法:加入宏包\verb|\usepackage{float}|,并在\verb|\begin{figure}|后加上[H]属性,强制控制浮动位置。即\verb|\begin{figure}[H]|


\subsection{改变文字颜色} 
加入宏包\verb|\usepackage{xcolor}|,使用 \verb|\textcolor{color}{想输入的文字}| 控制颜色。

\subsection{列表} 
\subsubsection{无序列表} 
\textbf{代码:}
\begin{lstlisting}
    \begin{itemize}
        \item
        \item
    \end{itemize}
\end{lstlisting}
\qquad \textbf{例:}
\begin{itemize}
    \item 项1
    \item 项2
\end{itemize}

\subsubsection{有序列表} 
\textbf{代码:}
\begin{lstlisting}
    \begin{enumerate}
        \item
        \item
    \end{enumerate}
    \end{lstlisting}
\qquad \textbf{例:}
\begin{enumerate}
    \item 项1
    \item 项2
\end{enumerate}

\subsection{表格} 
\begin{lstlisting}
    \begin{tabular}{|p{3cm}|c|c|}
        \hline
        单元格1 & 单元格2 & 单元格3 \\
        \hline  \hline
        单元格4 & 单元格5 & 单元格6 \\
        \hline
        单元格7 & 单元格8 & 单元格9 \\
        \hline
    \end{tabular}
\end{lstlisting}
\begin{tabular}{|p{3cm}|c|c|}
    \hline
    单元格1 & 单元格2 & 单元格3 \\
    \hline  \hline
    单元格4 & 单元格5 & 单元格6 \\
    \hline
    单元格7 & 单元格8 & 单元格9 \\
    \hline
\end{tabular}\bigskip

\textbf{解释:}\\
1.\verb|p{3cm}|设置列宽\\
2.\verb|\hline|加横线\\
3.|p{3cm}|c|c|中的c表示该列居中对齐(centering)



\subsection{公式}
\subsubsection{行内公式}
代码:\verb|$此处填写公式$|

例:$F=ma$ , $E=mc^2$
\subsubsection{单独一行的公式}
有两种写法:

\textbf{写法一:}
\begin{lstlisting}
    \begin{equation}
        此处填写公式
    \end{equation}
\end{lstlisting}
\qquad 例:
\begin{equation}
    F=ma
\end{equation}

\textbf{写法二:}
\begin{lstlisting}
    \[
        此处填写公式
    \]
\end{lstlisting}
\qquad 例:
\[
    E=mc^2
\]

\newpage
\section{Git}
\end{document}
